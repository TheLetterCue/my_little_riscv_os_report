% The introduction sets the scene for the project and its aims and explains these very well and points at an evaluation strategy
\chapter{Introduction}
\label{cha:intro}
There are many low level concepts that are discussed over the course of a standard undergraduate degree. This project sets out to explore some these concepts through basic implementation and experimentation, using simple hardware to allow full comprehension of a system being used during development. This chapter will introduce the aims of the project, and describe the motivations behind the project.
\section{Aims}
The overall aim of this project is to produce an operating system for a RISC-V based microcontroller, that allows the execution of multiple processes. To do this, a RISC-V machine must be selected and a toolchain to develop, load, and execute/debug code on the machine must be made. The elements of the operating system will be split into three main aims; the execution and scheduling of processes, the implementation of basic memory management to restrict each processes memory access, and the development of \ac{io} to allow interaction between the processes and a user. The effectiveness and limitations of the system will then be evaluated and the RISC-V architecture will be compared with other \ac{risc} architectures.

\section{Motivation and Challenges}
The basis of this project is to facilitate experimentation with operating system concepts as, while in a university level course the concepts are explained, it is often not possible to showcase these concepts in a practical environment, limited often to simulations. By using a bare metal system, this project will allow full access to a machine with no other system software, allowing complete control over the environment, which allows the implementation of concepts like processes and memory management. \\
This has also been used as an opportunity to explore the RISC-V architecture; as the prevalent architecture used in microcontroller systems are ARM based, so the use of RISC-V in this project will enable the comparison of the two architectures, and evaluate the effectiveness of RISC-V as an architecture.\\
Since the project is being developed on a bare metal system, considerations like the toolchain used to develop and debug code are important as they will have to be configured for the specific system, which may be more challenging as the userbase of RISC-V is more limited, so the tools required will be both less documented and available. The loading of code onto the machine is also important to consider, as it is easily possible that faulty or buggy builds could cause serious malfunction in the machine, so the loading and debugging of code must be done with care.\\
Other challenges include system limitations, as unlike a modern machine that would be used for general use, a microcontroller will have vastly limited resources in comparison, resulting in a need for greater optimization and careful planning, so the requirements of the system do not exceed the resources of the machine. Another limitation is the machine \ac{io} capabilities. This limitation does not only affect the end result of the systems \ac{io}, but also increases the difficulty of development, as some features such as time based interrupts can be difficult to develop without physical feedback from the machine.


\section{Report Outline}
\begin{itemize}
    \item Introduction: This chapter gives the basic overview of the planned project
    \item Background: This chapter will introduce key concepts to enable a more comprehensive understanding of the projects details
    \item Design: This chapter will outline the key design decisions made before and during the project
    \item Implementation: This chapter will give details on the practical implementation of concepts and designs mentioned in previous chapters
    \item Evaluation: This chapter will describe how the project was tested for completeness and correctness
    \item Conclusion: This chapter will give a summary of how the aims of the project were met 
\end{itemize}

