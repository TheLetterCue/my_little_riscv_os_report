% The introduction sets the scene for the project and its aims and explains these very well and points at an evaluation strategy
\chapter{Introduction}
\label{cha:intro}
There are many low level concepts that are discussed over the course of higher education. This project sets out to explore some these concepts through basic implementation and experimentation, using simple hardware to allow full comprehension of the system being used during development. This chapter will introduce the aims of the project, and describe the motivations behind the project.
\section{Aims}
The overall aim of this project is to produce an operating system for a \gls{riscv} based microcontroller, that allows the execution of multiple processes. To do this, a RISC-V machine must be selected, and a toolchain to develop, load, and execute/debug code on the machine must be made. The elements of the operating system will be split into three main aims, the exectution and scheduling of processes, the implementation of basic memory management to restrict each processes memory access, and the development of \ac{io} to allow interaction between the processes and a user. The effectiveness and limitations of the system will then be evaluated, and the RISC-V architecture will be compared with other \ac{risc} architectures.

\section{Motivation and Challenges}
This project is an exploration of concepts and methods of operating systems, with the overall goal to implement these concepts to create a simple operating system. 
In addition to this, the project will be done using a board that implements the open source, RISC-V processor architecture, which will require research into the operation of RISC-V, as well as any implementation specific features to the board in use. The board used will be the Sifive Hifive1 Rev B.
The main goal is the implementation of an interactive scheduler to run multiple processes at once, as this is something that will not be implemented unless part of an operating system.

\section{Project Plan}
The initial part of this project will be detemining an appropriate microcontroller board and setting up the toolchain required to compile and load code onto the board. Following this, the basic components of the system will be set up, such as the interrupt handler. Simple \ac{io} will be setup, to allow for ease of debugging, which will then be reimplemented later to be consistent with other IO elements. The implementation of processes will be split into multiple parts, such as the storage of process information, the creation of processes, the scheduling of processes and the deletion of processes. As part of this, memory management will be developed at the same time, as a basic implementation is required for user code to function.

\section{Report Outline}
\begin{itemize}
    \item Introduction: This chapter gives the basic overview of the planned project
    \item Background: This chapter will introduce key concepts to enable a more comprehensive understanding of the projects details
    \item Design: This chapter will outline the key design decisions made before and during the project
    \item Implementation: This chapter will give details on the practical implementation of concepts and designs mentioned in previous chapters
    \item Evaluation: This chapter will discuss the implementation and findings made during the implementation
    \item Conclusion: This chapter will give a summary of how the aims of the project were met 
\end{itemize}

